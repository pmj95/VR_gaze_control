%!TEX root = ../dokumentation.tex


\newcommand{\deAbstractContent}{
    Durch immer stärker werdende Computer Hardware und sinkende Preise bei Virtual Reality (VR) Systemen wird VR immer beliebter. Bisherige Eingabemethoden beschränken sich hauptsächlich auf die mitgelieferten Controller. Jedoch gibt es mittlerweile einige Firmen, die Einbaulösungen für Eye-Tracking in VR-Brillen anbieten, einige VR Brillen kommen bereits mit Eye-Tracking installiert. Die Anwendungsgebiete sind sehr zahlreich. In dieser Arbeit wird untersucht, ob Eye-Tracking als Eingabemethode geeignet ist. Dafür wurde eine Versuchsumgebung erstellt, in der zwei Zielmethoden, der klassische Laserpointer mit dem VR-Controller und Eye-Tracking, mit zwei Bestätigungsmethoden, dem Trigger des Controllers und Blinzeln, verbunden wurde. Dadurch entstehen vier Steuermethoden, die in Probandenversuchen mit vier Testpersonen getestet wurden. Der Proband muss Knöpfe in einer bestimmten Reihenfolge betätigen. Jede Steuermethode wird jeweils einmal mit großen und kleinen Knöpfen durchgeführt. Dabei werden neben dem subjektiven Empfinden auch die Genauigkeit und die Zeiten der Versuche untersucht. Dabei stellte sich die herkömmliche Kombination aus Laserpointer und Trigger als die effizienteste (schnellste und genauste) Methode heraus. Blinzeln als Bestätigungsmethode in Kombination mit dem Laserpointer konnte ähnliche Werte in diesen Kategorien erreichen. Eye-Tracking mit Trigger landete auf dem dritten Platz, konnte aber mit guten Werten, vor allem bei großen Knöpfen, überzeugen. Eye-Tracking mit Blinzeln hat mit Abstand am schlechtesten abgeschlossen und ist, in der implementierten Form, keine valide Möglichkeit zur Steuerung in VR. Ein weiterer Versuchsaufbau, basierend auf den hier erlangten Erkenntnissen, können die Eignung von Eye-Tracking zur Steuerung in größeren Probandenversuchen und angepasstem Versuchsaufbau besser darstellen. Die hier beschriebenen Ergebnisse deuten darauf hin, dass Eye-Tracking eine valide Option ist.
    
    Das komplette Projekt ist auf GitHub unter folgendem Link zu finden: \url{https://github.com/pmj95/VR_gaze_control}
}

\newcommand{\enAbstractContent}{
    Increasing power in computer hardware and more competitive pricing in virtual reality (VR) lead to more and more popularity for VR systems. The input most widely used are the physical controllers coming with the VR system. However, there are more and more providers of eye tracking, some providing add-on solutions, some providing built-in solutions for eye tracking in VR headsets. Eye tracking can be used in a variety of ways in VR. In this paper, the possibility of using eye-tracking as an input method is investigated. A VR environment is created, in which varying input methods are compared. For aiming the standard VR approach of a laser pointer using the physical controllers and eye tracking are compared. For input selection blinking and the trigger of the controller are used. This results in four different input methods. The input methods are compared in a trial with four participants. The study participant needs to select a pre-defined sequence of buttons with each input method. Each input method is tested once with small and once with big buttons. To compare the different input methods, the subjects are asked about their experience and data like the accuracy and the times to complete the tasks are measured and analyzed. The combination laser pointer/trigger with the physical controller proofed to be the most effective way of navigating the test level. Using blinking instead of the trigger in combination with the laser pointer proofed to be almost as effective. Eye tracking used with the trigger of the controller showed to be a viable way of navigating the level but fell short of the two variants using the laser pointer. The combination of eye tracking and blinking proofed far worse than all other input methods and should, in the current implementation, not be used for controlling environments in VR. This paper is meant as a reference point for future studies using eye tracking as input method. The results are promising, although further research is required. 
    
    The whole project can be found on GitHub under \url{https://github.com/pmj95/VR_gaze_control}
}

%%%%%%%% Ab hier nicht mehr anfassen! %%%%%%%%

\newcommand{\deAbstract}{%
    \renewcommand{\abstractname}{\langabstract} % Text für Überschrift
    \begin{abstract}
        \thispagestyle{plain}
        \deAbstractContent
    \end{abstract}
}

\newcommand{\enAbstract}{
    \renewcommand{\abstractname}{\langabstract} % Text für Überschrift
    \begin{abstract}
        \thispagestyle{plain}
        \enAbstractContent
    \end{abstract}
}

\iflang{de}{
    \deAbstract
    \ifbothabstracts
        \clearpage
      %  \begin{otherlanguage}{english}
            \enAbstract
     %   \end{otherlanguage}
    \fi
}

\iflang{en}{
    \enAbstract
    \ifbothabstracts
        \clearpage
        \begin{otherlanguage}{ngerman}
            \deAbstract
        \end{otherlanguage}
    \fi
}