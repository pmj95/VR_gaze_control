%!TEX root = ../dokumentation.tex
\chapter{Einleitung}
Der Markt von Virtual Reality Systemen ist seit Jahren am Expandieren. Das liegt vor allem an einer höheren Nachfrage, immer reicher werdende Käufer in reicher werdenden Ländern und einer hohen Konkurrenz in der Industrie \cite{grandview.2020}. Ein Hauptziel der Erfahrungen in \ac{VR} ist eine möglichst hohe Immersion in die virtuelle Welt. Vor allem die Gaming Industrie ist hier ein treibender Faktor, aber auch für soziale Plattformen oder in der Wissenschaft findet \ac{VR} eine Anwendung \cite{grandview.2020}. Derzeit werden hauptsächlich extern getrackte Controller für die Interaktion in \ac{VR} verwendet. Das erlaubt bereits immersive Interaktionen mit der virtuellen Welt, allerdings erlauben die Controller nur eine festgesetzte Anzahl an unterschiedlichen Aktionen. So ist der Controller der HTC Vive beispielsweise auf digitale Knöpfe, ein Touchpad, einem Trigger und die Bewegung des Controllers limitiert \cite{ViveProduct}. Valve hat mit dem Index Controller zwar noch eine Möglichkeit des Finger-Trackings in einen Controller integriert, jedoch ist auch hier immer noch nur eine Interaktion mit den Händen möglich \cite{Index.Controller}. Für eine volle Immersion fehlen daher noch Informationen, wie etwa die Bewegung der Füße, die Mimik, oder die Bewegung der Augen. Gerade beim Eye-Tracking gibt es derzeit viele unterschiedliche Ansätze, die langsam ihren Weg auf den Markt schaffen. Dabei kann Eye-Tracking sehr vielseitig eingesetzt werden:
\begin{itemize}
	\item Eine Verringerung der Rechenleistung durch das Erkennen des aktiv betrachteten Bereichs. Dadurch ist es möglich, nur diesen Teil des Bildes scharf darzustellen, ohne dass dies dem Nutzer auffällt. \cite{Rogers.2019} (vgl \autoref{section:foveated})
	\item Natürlichere Interaktionen. Einerseits kann in Multiplayer-Umgebungen die Bewegung der Augen aufzeigen und diese an dem virtuellen Charakter anzeigen. Andererseits kann Eye-Tracking zur Interaktion mit der Umgebung genutzt werden.\cite{Rogers.2019} 
	\item Besseres Verstehen des Verhaltens der Nutzer. Blickdaten können ausgewertet werden, um Rückschlüsse für beispielsweise das Design von Fluchtwegen zu ziehen oder Marketing und Design zu überprüfe. \cite{Rogers.2019}
\end{itemize}
In dieser Arbeit wird Eye-Tracking als Steuermöglichkeit untersucht. Die konkrete Frage, die beantwortet werden soll, lautet: \glqq Ist Eye-Tracking für eine Steuerung in \ac{VR} geeignet?\grqq{} Zur Beantwortung dieser Frage wurde ein Probandenversuch mit 20 Teilnehmern konzeptioniert. Hierbei werden vier unterschiedliche Steuermethoden miteinander verglichen, um eine Aussage über die Eignung von Eye-Tracking im Vergleich zu anderen Steuermethoden treffen zu können. Die Steuermethoden setzen sich aus jeweils zwei Ziel- und Auswahlmethoden zusammen. Zum Zielen werden ein Laserpointer mit dem klassischen \ac{VR} Controller und Eye-Tracking verwendet. Zur Auswahl werden der Trigger des Controllers und Blinzeln genutzt. Daraus ergeben sich vier Kombinationsmöglichkeiten. Des Weiteren werden die Distanz der Knöpfe zum Spieler und die Größe der Knöpfe variiert, um die Auswirkung dieser beiden Parameter untersuchen zu können. Während der Versuche werden Daten, wie etwa der Dauer pro Versuch, der Dauer zwischen dem Auswählen zweier Knöpfe, den Blickdaten und das subjektive Empfinden der Probanden aufgezeichnet. Mit einer Analyse der Daten wird eine begründete Antwort auf die Forschungsfrage erarbeitet. Da zum geplanten Zeitpunkt der Probandenversuche die Ausgangsbeschränkungen aufgrund der \ac{COVID-19}-Pandemie in Kraft waren, konnte die Studie nicht im vollen Rahmen durchgeführt werden. Deshalb wurde die Teilnehmerzahl auf vier Probanden, wovon zwei die beiden Autoren dieser Arbeit sind, reduziert und die Anzahl der Versuche auf die wesentlichen Aspekte reduziert. Das bedeutet, dass die Distanz zu den Knöpfen gar nicht und die Größe der Knöpfe nur auf zwei Stufen variiert wurde. Daraus ergeben sich acht mögliche Versuchsaufbauten, jede Steuerart je einmal mit großen und einmal mit kleinen Knöpfen. Trotz dieser Einschränkung ist eine Beantwortung der Forschungsfrage möglich, die geringe Probandenzahl und der geänderte Versuchsaufbau müssen bei der Interpretation der Ergebnisse jedoch berücksichtigt werden. 
