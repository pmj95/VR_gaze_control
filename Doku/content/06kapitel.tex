%!TEX root = ../dokumentation.tex

\chapter{Diskussion der Ergebnisse}
\label{section:discussion}
In diesem Abschnitt werden die in \autoref{section:results} Ergebnisse diskutiert \todo{JH: Der Satzbau passt nicht: werden die Ergebnisse aus Abschnitt 5.4 diskutiert}. Diese Trennung erfolgt, um die Ergebnisse der Studie in \autoref{section:results} ohne Wertung darstellen zu können und in diesem Abschnitt eine Bewertung dieser Ergebnisse \todo{JH: und um eine Bewertung dieser Ergebnisse in diesem Abschnitt} vornehmen zu können \todo{JH: zu können doppelt sich; besser: darzustellen, vorzunehmen}, so dass die Ergebnisse möglichst neutral dargestellt werden. Wie in \autoref{section:corona} schon erläutert, sind die Ergebnisse aufgrund der COVID-19 Pandemie nur bedingt aussagekräftig. Dieser Abschnitt wird jedoch trotzdem die gegebenen Daten auf Indizien untersuchen, ob Eye-Tracking eine valide Methode zur Steuerung in VR ist. 

\section{Bewertungskriterien}
\label{section:criteria}
Um bewerten zu können, ob eine Steuermethode zur Steuerung in VR geeignet ist, müssen Kriterien festgelegt werden, die eine Bewertung dieser Methode zulassen. Die im Folgenden diskutierten Steuermethoden müssen demnach eine Genauigkeit von über 75\% in den Versuchen und eine durchschnittliche unter 15 Sekunden erreichen, um als gute Steuermethode zu gelten. Ab einer Genauigkeit von 90\% und einer Gesamtdauer von unter 10 Sekunden gilt eine Steuermethode als sehr gut. \todo{CM: bisschen kurz und benötigt evtl ne Quelle oder sowas}

\section{Verbindung der unterschiedlichen numerischen Werte}
Da zwischen mehreren unterschiedlichen Metriken, wie beispielsweise der Gesamtdauer und der Genauigkeit, eine Korrelation erkennbar ist, ist es sinnvoll, zu überprüfen, ob diese auch die selbe Ursache haben. Diese Korrelationen werden in diesem Abschnitt betrachtet, um mögliche Ursachen zu finden.

\subsection{Zusammenhang von Gesamtdauer und Genauigkeit eines Versuchs}
In \autoref{section:comparison} wird die Gesamtdauer und die Genauigkeit der unterschiedlichen Steuermethoden für große Knöpfe betrachtet. Hierbei schneidet die Steuermethode Laserpointer/Trigger in der Gesamtdauer am besten ab, dicht gefolgt von der Steuermethode Laserpointer/Blinzeln. Beide Steuermethoden haben bei großen Knöpfen eine höhere Genauigkeit verglichen mit den Eye-Tracking-basierten Methoden. Bei der Steuermethode Eye-Tracking/Trigger sinkt die Genauigkeit im Vergleich zu den Versuchen mit dem Laserpointer. Die Gesamtdauer ist leicht angestiegen. Wenn Eye-Tracking mit Blinzeln benutzt wird, ist die Genauigkeit signifikant schlechter. Hier hat nur Proband 2 eine Genauigkeit von über 75\% erreicht, alle anderen Probanden lagen sogar unter 50\%. Die Gesamtdauer ist hier deutlich angestiegen. Es besteht also ein klarer Zusammenhang zwischen Genauigkeit und Gesamtdauer. 

Bei den Zielmethoden scheint der Laserpointer dem Eye-Tracking leicht überlegen zu sein. Dies ist vor allem an den Versuchen Laserpointer/Trigger und Eye-Tracking/Trigger zu erkennen. Hier ist die Genauigkeit beim Laserpointer mit durchschnittlich 95,83\% deutlich höher, als beim Eye-Tracking mit durchschnittlich 75,35\%. Allerdings ist hier wichtig zu beachten, dass Proband 2 und Proband 3 die Autoren dieser Arbeit sind und somit schon mehr Erfahrung mit der Steuerung über Eye-Tracking sammeln konnten. Erfahrung scheint hier also ein wichtiger Faktor in der Genauigkeit zu sein. Dies spiegelt sich auch in der Gesamtdauer wieder, wo Proband 2 und 3 in beiden Versuchen am besten abgeschnitten haben. Im Versuch Laserpointer/Trigger haben die drei \todo{JH: sind doch nur zwei? CM: Ne einer meiner kumpels hat auch VR daheim, und da es hier um den Laserpointer geht sind es drei}Probanden mit Erfahrung in VR, die über ein einfaches ausprobieren rausgeht, die besten Zeiten und Genauigkeiten erzielt. Einzig Proband 4, der bisher fast keine Erfahrungen mit VR machen konnte, hat hier schlechter abgeschnitten. Auch hier ist die Erfahrung also ein wichtiger Faktor, allerdings nicht so signifikant, wie beim Eye-Tracking. 

Blinzeln scheint als Auswahlmethode dem Trigger unterlegen zu sein. Mit dem Laserpointer war die Genauigkeit zwar noch ähnlich hoch, allerdings ist die Kombination aus Eye-Tracking und Blinzeln sowohl in der Gesamtdauer, als auch in der Genauigkeit signifikant schlechter. Eine durchschnittliche Genauigkeit von unter 50\% ist keinesfalls akzeptabel für eine Steuerung, bei der nur Knöpfe betätigt werden müssen. Auch die Probanden mit viel Erfahrung konnten hier keine guten Werte erzielen, einzig Proband 2 sticht mit einer Trefferquote von 71,43\% hervor. Die geringe Genauigkeit wird vor allem durch das Blinzeln während der Benutzung des Eye-Trackers hervorgerufen. Da der Eye-Tracker ohne das Blinzeln eine deutlich höhere Genauigkeit ermöglicht hat, muss das Blinzeln einen sehr starken Einfluss haben. Dabei kommt es vor allem zu dem Effekt, dass der Eye-Tracker keine verlässlichen Daten liefert, wenn die Augen teilweise geschlossen sind. Dadurch verrutscht der berechnete Punkt und die VR-Umgebung kann nicht mehr richtig erkennen wohin der Nutzer seinen Blick richtet. Dies hat zur Folge, dass der Knopf nicht ausgewählt wird. Dies ist sehr gut in den Heatmaps zu erkennen. Mehr hierzu in \autoref{section:heatmapsAnalysis}. 

\subsection{Einfluss der Veränderung der Knopfgröße}
Die Veränderung der Knopfgröße soll, unter Beachtung von Fitts' Gesetz, zu einer Veränderung der Zeit zum Aktivieren von Knöpfen führen. Jede Steuermethode wird je ein mal mit großen und kleinen Knöpfen untersucht. Dabei wird die Breite, beziehungsweise der Durchschnitt, der Knöpfe halbiert. Dadurch sollte die Zeit zum Aktivieren steigen, wenn die Knöpfe kleiner werden. Im Durchschnitt ist dies bei jedem Versuchsaufbau der Fall gewesen (vgl. \autoref{fig:timesIncrease}). Teilweise kam es vor, dass Probanden schneller waren, als im selben Versuch mit größeren Knöpfen. Dies ist meist auf einen Lerneffekt zurückzuführen. Dieser Effekt ist nur bei den Autoren, Probanden 2 und 3, aufgetreten, welche beide viel Erfahrung mit den Versuchsaufbauten hatten. Proband 1 und 4 haben in jedem Versuch mit kleineren Knöpfen länger gebraucht, als im selben Versuch mit größeren Knöpfen. Proband 1 liefert in den Versuchen mit kleinen Knöpfen die interessantesten Daten, da er weder Erfahrungen mit Eye-Tracking hatte noch, wie es bei Proband 4 der Fall war, Probleme mit dem Eye-Tracking hatte. In den Gesamtzeiten war der Anstieg immer in einem Bereich, der sehr gut mit der Vorhersage durch Fitts' Gesetz übereinstimmt. Bei Proband 4 musste der Eye-Tracker\todo{JH: glaube Eye-Tracker haben wir davor als zusammen geschrieben CM: lt wiki muss es mit bindestrich geschrieben} zwischen unterschiedlichen Versuchen mehrmals neu kalibriert werden, da immer wieder ein Drift aufgetreten ist (vgl \autoref{section:calibration}). Dies spiegelt sich in den teils enorm langen Zeiten und der sehr niedrigen Genauigkeit in den Versuchen mit Eye-Tracking wieder. 

Der Laserpointer erweist sich bei kleinen Knöpfen als die deutlich bessere Zielmethode. Hier waren die Zeiten, je nach Auswahlmethode, doppelt bis 2,5 mal so schnell wie bei Eye-Tracking mit Trigger zum Bestätigen. Die Genauigkeit ist im Durchschnitt höher, als bei den Versuchen mit Eye-Tracking. Der Unterschied zwischen der Genauigkeit beim Laserpointer mit Blinzeln im Vergleich zu Laserpointer/Trigger kommt vermutlich durch einen Lerneffekt zustande. Da jeder der Probanden zuerst die beiden Versuche mit Laserpointer/Trigger macht und danach die beiden Versuche mit Laserpointer/Blinzeln ist der Versuch Laserpointer/Blinzeln mit kleinen Knöpfen bereits der vierte Versuch mit dem Laserpointer. Die aus den ersten drei Versuchen erlangte Erfahrung ist ein Faktor, der die Genauigkeit beim letzten Versuch mit dem Laserpointer beeinflussen kann. Eye-Tracking erzielt bei dieser Größe Werte für Zeiten und Genauigkeit\todo{JH: entweder beide Singular oder Plural; bin mir aber auch nicht sicher CM: uff gute frage}, die nicht mehr den Anforderungen einer zuverlässigen Steuerung entsprechen (vgl. \autoref{section:criteria})

Bei der Bestätigungsmethode zeichnet sich ein sehr ähnliches Bild ab, wie bei den großen Knöpfen. Blinzeln ist in Verbindung mit dem Laserpointer eine valide Option, in Verbindung mit Eye-Tracking jedoch ist die Genauigkeit und die Gesamtdauer so schlecht, dass es als Steuermöglichkeit keine Option darstellt. Mit einer Genauigkeit von 14\% ist nur eine von sieben Eingaben korrekt, der Rest sind Fehleingaben, die eine Menüsteuerung erschweren oder sogar komplett unmöglich machen. Diese Ungenauigkeit und lange Dauer ist auch sehr gut in den Heatmaps, vor allem von Proband 4, im \autoref{appendix:heatmaps} \todo{JH: statt auf Abschnitt lieber direkt auf Abbildung verweisen CM: war hier Absicht, da es mehrere Abbildungen sind. Macht die Ref auf die Abbildung direkt dann trotzdem Sinn?} zu sehen.

\subsection{Verbindung zwischen Blickdaten und Dauer und Genauigkeit}
\label{section:heatmapsAnalysis}
Beim Betrachten der Blickdaten fällt auf, dass bei jedem Proband die Menge an Linien bei den Versuchen zu Eye-Tracking/Blinzeln deutlich höher ist, als bei den Versuchen mit Eye-Tracking/Trigger \todo{JH: Nicht konsequent genug. Ich würde die Steueroption mit / trennen. Da haben wir hoffentlich eine klare Linie in der Arbeit CM: Das muss ich dann mal noch durchgehen, kann gut sein dass ich hin und her wechsle ich lasse das todo hier mal als reminder drin}. Dies deutet auf eine deutlich höhere Gesamtzeit und deutlich höhere Zeit zwischen dem Auswählen der Knöpfe hin. Dies ist auch in den Grafiken zur Gesamtzeit (vgl. \autoref{appendix:timessmall} \autoref{fig:totalTimesBig}) und zur Zeit zwischen dem Betätigen der Knöpfe (vgl. \autoref{fig:totalTimesBig}) zu sehen. Dies liegt unter anderem an der Ungenauigkeit, die bei dem Blinzeln zum Bestätigen in der Steuerung auftritt. Bei großen Knöpfen ist diese Störung nicht so extrem, da ein leichtes Verrutschen der Blickdaten nicht direkt ein Abkommen vom Knopf bedeuten muss. Bei kleinen Knöpfen ist das deutlich wahrscheinlicher, da nur noch eine kleinere Distanz als Störung zurückgelegt werden muss, sodass der Knopf nicht mehr getroffen wird. Die Auswirkungen dieses Verhaltens sind in der Genauigkeit in den Abbildungen \ref{fig:totalACCBig} und \ref{fig:totalACCsmall} sehr gut zu erkennen. 

Die Distanz zwischen den Knöpfen spielt beim Eye-Tracking im genutzten Versuchsaufbau keine große Rolle mehr. Das ist in \autoref{fig:plotbuttons} zu erkennen. Hier sind die Werte bei den beiden Eye-Tracking Versuchsaufbauten nicht mit Fitts' Gesetz zu erklären. Nach Fitts müsste die Zeit zum Aktivieren der Knöpfe mit einer Erhöhung der Distanz ansteigen. Dieses Verhalten ist nur bei dem Versuchsaufbau Laserpointer/Trigger zu beobachten. Bei allen anderen Steuermethoden sind Faktoren, wie etwa der Lerneffekt oder die Gewöhnung an eine neue Steuermethode, ausschlaggebender. Bei den Versuchen mit Eye-Tracking kommt hinzu, dass mit den Augen nicht der direkte Weg zwischen zwei Knöpfen zurückgelegt wird, sondern auf der Anzeige ein Zwischenstopp eingelegt wird, um nachzuschauen, welcher Knopf als nächstes gedrückt werden soll. Damit ist die Distanz zwischen den Knöpfen nur noch zweitrangig, da zwei Knöpfe, in diesem Versuchsaufbau beispielsweise die Knöpfe 13 und 14, direkt nebeneinander liegen können. Der Proband muss so erst ganz nach oben schauen, um zu sehen, dass Knopf 14 als nächstes gedrückt werden muss. Dadurch ist eine Analyse der Zeit zwischen zwei Knöpfen in Abhängigkeit der Distanz in diesen Versuchen nicht aussagekräftig.

\subsection{Einordnung der Ergebnisse des Fragebogens}
Die Ergebnisse aus der Befragung kommen zu dem selben Ergebnis, wie die Analyse der aufgezeichneten Werte. Im Ranking bewerteten alle Probanden die Methode Laserpointer/Trigger am besten und Eye-Tracking/Blinzeln am schlechtesten. Die Probanden 1, 2 und 3 haben die Methode Laserpointer/Blinzeln am zweitbesten und Eye-Tracking/Trigger am drittbesten bewertet. Proband 4 hatte diese beiden Steuermethoden in umgekehrter Reihenfolge. Im Durchschnitt sind die Werte im Ranking also übereinstimmend mit der Analyse der Gesamtzeit und der Genauigkeit bei den unterschiedlichen Versuchen. Interessanterweise hat Proband 4, bei dem Probleme mit dem Eye-Tracking aufgetreten sind, die Methode Eye-Tracking/Trigger als zweitbeste Methode wahrgenommen. Das Blinzeln scheint hier ein größerer Störfaktor gewesen zu sein, als das Eye-Tracking selbt. 

\subsection{Einfluss der Verkleinerung der Knöpfe}
Auch im Anbetracht der Verkleinerung der Knöpfe sind die Ergebnisse mit den Zahlen übereinstimmend. Alle Probanden gaben an, dass eine Verkleinerung die Fähigkeit die Knöpfe auszuwählen beeinflusst hat. Der angegebene Einfluss deckt sich größtenteils mit dem Anstieg in Gesamtdauer und der Abnahme in der Genauigkeit bei gleichen Steuermethoden mit unterschiedlicher Größe (vgl. \autoref{section:influencebuttonsize} und \autoref{section:resultquestions}). Auch die Angaben zur Fähigkeit, Knöpfe in den Versuchen mit Eye-Tracking anzuvisieren und auszuwählen, decken sich größtenteils mit den Heatmaps im \autoref{appendix:heatmaps}. Auch bei den Versuchen mit dem Laserpointer sind die Daten aus der Befragung mit den Daten aus \autoref{section:comparison} übereinstimmend. 

Die subjektive Empfindung spiegelt sich insgesamt in den objektiv gemessenen Werten wieder. Die Ergebnisse aus der Analyse der Zahlen, dass die Kombination Laserpointer/Trigger am Besten funktioniert ist hier auch gegeben. Die Methoden Laserpointer/Blinzeln und Eye-Tracking/Trigger bieten jedoch beide eine sehr gute Alternative. Nur einer der Nutzer kann sich eine Steuerung mit Laserpointer/Blinzeln gar nicht vorstellen. Die Kombination aus Eye-Tracking und Blinzeln ist bei keinem der Probanden beliebt und keiner kann sich eine Steuerung hiermit vorstellen.

\section{Zusammenfassung der Ergebnisse \& Beantwortung der Forschungsfrage}
Insgesamt stellt die Steuermethode Laserpointer/Trigger die beliebteste und zuverlässigste Eingabemethode dar. Allerdings ist Eye-Tracking auch eine valide Steuermethode, solange es nicht mit Blinzeln zum Bestätigen kombiniert wird. Eye-Tracking in Kombination mit einer alternativen Bestätigungsart, wie etwa einem Trigger, ist eine gute Möglichkeit Eye-Tracking in VR als Steuermöglichkeit zu integrieren. Damit kann eine zusätzliche, immersive Steuermethode genutzt werden, mit der sich neue Möglichkeiten der Interaktion eröffnen. Auch für körperlich eingeschränkte Menschen kann dies von enormen Vorteil sein, da sich so eine neue Möglichkeit der Interaktion eröffnet und so die Welt der Virtuellen Realität für mehr Menschen zugänglich wird. \todo{JH: Also das passt alles. Finde nur iwie ist die Beantwortung etwas kurz oder nicht? Da könnten wir evtl. auch Reichardt mal fragen :D CM: Jo wir fragen ihn wie lang diese Beantwortung sein muss}

\section{Einordnung in die Literatur}
\citeauthor{Pai.2019} sind in ihrer Untersuchung zu dem Schluss gekommen, dass Eye-Tracking die effektivste Steuermethode in VR ist. Zu diesem Schluss kommt diese Arbeit nicht, da sowohl in der Befragung, als auch in den gemessenen Werten die Steuerung Laserpointer/Trigger überlegen war. Allerdings sind die Versuche nur mit einer sehr kleinen Gruppe von vier Personen durchgeführt worden. Außerdem sind die Ergebnisse sehr knapp, weshalb es möglich ist, dass mit einer höheren Probandenzahl ein anderes Ergebnis auftreten kann. Die Ergebnisse der beiden Untersuchungen sind  sehr vergleichbar, da das Eye-Tracking mit Trigger in den Probandenversuchen nur knapp hinter dem Vive Controller landete. Ob der Nutzer einen Trigger an einem Controller betätigt, oder ob er seinen Arm anspannt, sollte keinen allzu großen Unterschied machen. In dem Schluss, dass Eye-Tracking eine valide Steuermethode darstellt, können die Autoren \citeauthor{Pai.2019} zustimmen. Auch die Untersuchung von \citeauthor{D.Kumar.2016} kommt zu dem Schluss, dass Eye-Tracking als Eingabemethode in VR geeignet ist. \cite{D.Kumar.2016} \cite{Pai.2019}
\todo{JH: Scheint mir auch etwas kurz??? Aber da evtl. nochmal Herr Reichardt fragen :D}



	
