%!TEX root = ../dokumentation.tex

\chapter{Grundlagen}
In dem folgenden Kapitel werden die Grundlagen dieser Arbeit beschrieben. Zuerst wird eine Definiton für \acl{VR} festgelegt. Danach werden die Headsets für \acl{VR} und für das Eye-Tracking vorgsetellt. Am Ende der Grundlagen folgt eine Einführung in die Laufzeit- und Entwicklungsumgebung Unity.


\section{\acl{VR}}
Für den Begriff \ac{VR} oder Virtuelle Realität existiert keine einheitliche Definition. Für \citeauthor{8070720} ist \ac{VR} 
eine wissenschaftliche Technologie für die Mensch-Maschine-Interaktion zum Verstehen und Simulieren einer realen oder virtuellen Umgebung. 
der Mensch-Maschine-Interaktion zum Verstehen und Simulieren der realen oder virtuellen Umgebung in einem 3D-Raum, die der realen Umgebung sehr ähnlich ist und die Natur in einem bestimmten Bereich erlebt, indem sie auftaucht und physische Präsenz in dieser Umgebung zeigt, indem Hochleistungscomputer zusammen mit sensorischer Ausrüstung wie Handschuhen, Kopfhörern usw. verwendet werden.\grqq \cite[vgl.]{8070720}

%Virtuelle Realität ist eine wissenschaftliche Technologie der Mensch-Maschine-Interaktion zum Verstehen und Simulieren der realen oder virtuellen Umgebung in einem 3D-Raum, die der realen Umgebung sehr ähnlich ist und die Natur in einem bestimmten Bereich erlebt, indem sie auftaucht und physische Präsenz in dieser Umgebung zeigt, indem Hochleistungscomputer zusammen mit sensorischer Ausrüstung wie Handschuhen, Kopfhörern usw. verwendet werden.

% relativ gutes Buch https://link.springer.com/book/10.1007/978-3-662-58861-1
% Unterteilen in Definition, Technologie / Einsatzgebiete und evtl. in kritische Würdigung

\section{\acs{VR}-Headset}
Im Rahmen dieser Arbeit wird das \acs{VR}-Headset HTC Vive verwendet. Das Headset wurde gemeinsam von den Firmen HTC und Valve entwickelt. Das \acs{VR}-Headset lässt sich in die folgenden 3 Teile unterteilen. 

\cite{Clay_Koenig_Koenig_2019}
\subsection{\acl{HMD}}
\ac{HMD}

\begin{figure}[!htbp]
	\centering
	\includegraphics[width=1\linewidth]{vive-hardware-hmd-1}
	\caption[HTC Vive HMD]{HTC Vive \acs{HMD} \cite{ViveHMD}}
	\label{fig:vive-hardware-hmd-1}
\end{figure}

\todo{Abbildung von Headset}

\subsection{Controller}
\begin{figure}[!htbp]
	\centering
	\includegraphics[width=1\linewidth]{vive-hardware-controllers-1}
	\caption[Controller der HTC Vive]{Controller der HTC Vive \cite{ViveControllers}}
	\label{fig:vive-hardware-controllers-1}
\end{figure}
\todo{Abbildung von Controller}

\subsection{Lighthouse Tracking System}
\begin{figure}[!htbp]
	\centering
	\includegraphics[width=1\linewidth]{vive-hardware-base-stations}
	\caption[Lighthouse Basis Station]{Lighthouse Basis Station \cite{ViveBaseStation}}
	\label{fig:vive-hardware-base-stations}
\end{figure}
\todo{Abbildung von Basisstation}

\section{Eyetracking}
Eyetracking ist eine Technologie, die erkennt, in welche Richtung eine Person ihren Blick richtet. Hierfür werden beim Eyetracking Blick sowie Augenbewegungen erfasst. Die hauptsächlichen Parameter, die durch das Eyetracking erfasst werden, sind Fixiationen (von dem Benutzer fixierter Punkt oder Objekt), Sakkaden (schnelle Augenbewegungen bei der Erfassung eines neuen Fixpunktes) und Regression (Rücksprung zu vorherigen Fixpunkte oder Objekte).

\begin{figure}[!htbp]
	\centering
	\includegraphics[width=1\linewidth]{pupil_labs_headset}
	\caption[Pupil Core Headset]{Pupil Core Headset}
	\label{fig:pupil_labs_headset}
\end{figure}

\todo{Quelle aus Pupil Labs Dokumentation --> Core --> Hardware}

Im Rahmen dieser Arbeit wird ein Eyetracker von Pupil Labs verwendet. Sie entwickeln seit 2014 die Plattform Pupil Core, die aus einer Open-Soure-Suite sowie dem Eyetracker selber besteht. In \autoref{fig:pupil_labs_headset} ist das tragbare Eyetracker Headset Pupil Core zu sehen, welches wie eine Brille getragen wird. An dem Headset ist eine Blickfeldkamera (Nummer 1) angebracht, welche das Blickfeld des Benutzers aufnimmt. Mithilfe der Augenkameras (Nummer 3) lässt sich das komplette Auge erfassen. Die Augenkameras lassen sich individuell auf die Augen einstellen. Ein USB-C Kabel (Nummer 4) dient als Stromversorgung sowie für den Austausch der Videodaten der Kameras. Wird der Eyetracker mit dem Computer verbunden, dann lässt sich mit der Pupil Core Software das Eyetracking starten. Mithilfe der Software lassen sich die Eyetracking-Daten aus den Videostreams auslesen, auswerten und über eine Netzwerk Schnittstelle zur Verfügung stellen. Zudem kann in das Umgebungsvideo der Blickfeldkamera der Punkt angezeigt werden, auf den der Benutzer seinen Blick fixiert. \\
Da in dieser Arbeit das Eyetracking innerhalb einer \ac{VR}-Umgebung untersucht werden soll, wird ein speziell für die HTC Vive entwickelter Eyetracker von Pupil Labs verwendet. Dies ist ein Add-on von Pupil Labs, welches in das HTC Vive Headset eingebaut wird und während dem Tragen des Headsets verwendet werden kann.

\todo{Bild von Add-on hinzufügen}

\cite{PaperPupilLabs}

\section{Unity}

\subsection{steamVR}

\subsection{hmd\_eyes}

\section{Fitts' Law}