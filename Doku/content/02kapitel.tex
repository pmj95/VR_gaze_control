%!TEX root = ../dokumentation.tex

\chapter{Grundlagen}
\todo[inline]{Grundlagen-Einleitung überarbeiten }
In dem folgenden Kapitel werden die Grundlagen dieser Arbeit beschrieben. Zuerst wird das in dieser Arbeit verwendete \acs{VR}-Headset vorgestellt. Nach dem \acs{VR}-Headset folgt der Eye Tracker und zum Schluss die Laufzeit- und Entwicklungsumgebung Unity.

\section{Virtuelle Realität}
Die Grundidee der virtuellen Realität existiert bereits seit dem 19. Jahrhundert. David Brewester entwickelte 1848 ein linsenförmiges Stereoskop, welches das Gefühlt von Tiefe und Immersion hervorbrachte. Nennenswerte Entwicklungen im Bereich der virtuellen Realität wurden zu Beginn des 20. Jahrhunderts getätigt. \cite[S. 1f]{Singh.2017} 1987 gebrauchte der Informatiker Jaron Lanier als erster Wissenschaftler den Begriff \ac{VR} \cite{Doerner2019}. 

Bei der Suche einer einheitlichen Definition fällt eine große Diskrepanz in der Literatur auf. Die \citeauthor{BundeszentralefurpolitischeBildung.2018} definiert \ac{VR} als \glqq Darstellung und gleichzeitige Wahrnehmung der Wirklichkeit und ihrer physikalischen Eigenschaften in einer in Echtzeit computergenerierten, interaktiven virtuellen Umgebung\grqq  \cite{BundeszentralefurpolitischeBildung.2018}. \\
\citeauthor{LaValle.2019} von der University of Oulu definiert \ac{VR} als herbeiführen eines gezielten Verhaltens in einem Organismus durch künstliche sensorische Stimulation, während der Organismus sich der Störung kaum oder gar nicht bewusst ist \cite[S. 1]{LaValle.2019}. Aus seiner Definition kristallisiert \citeauthor{LaValle.2019} die folgenden vier Hauptkomponenten:\cite[S. 1,3]{LaValle.2019} \todo{Seitenzahlen weg machen???}
\begin{enumerate}
	\item \textbf{Gezieltes Verhalten}: Der Organismus hat eine Erfahrung (zum Beispiel Fliegen), welche durch den Entwickler erstellt wurde. 
	\item \textbf{Organismus}: Organismus bezieht sich hierbei auf den Benutzer von \ac{VR}, welcher jeglicher Lebensform entsprechen kann. Wissenschaftler setzten die \ac{VR}-Technologie bereits bei Fruchtfliegen, Kakerlaken, Fischen, Nagetiere und Affen ein. 
	\item \textbf{Künstliche sensorische Stimulation}: Der Einsatz moderner Technik ermöglicht das Replizieren verschiedener sensorischer Erlebnissen und das Ersetzen durch den Einsatz künstlicher Simulationen. 
	\item \textbf{Bewusstsein}: Bei einem \acl{VR} Erlebnis wird der Organismus entsprechend getäuscht, sodass sich dieser im Unterbewusstsein sowohl wohl als auch präsent in der virtuellen Welt fühlt.
\end{enumerate}

Die Mitgründer von der \citename{omnia.2017}{editor} \citename{omnia.2017}{author} definierten in ihrer Masterarbeit \ac{VR} als \glqq ein Medium, welches aus einer computergenerierten, interaktiven Welt besteht, die den Nutzer vollständig umgibt und durch die Ansprache ein oder mehrerer Sinne mittels geeigneter Systeme besonders immersiv erlebt werden kann\grqq \cite{omnia.2017}. Sie haben dabei nicht nur andere Definitionen betrachtet, sondern auch die Bedeutung der Wörter virtuell und Realität. Virtuell wird im Duden als \glqq nicht echt \grqq \cite{DudenVirtuell} oder als \glqq nicht in Wirklichkeit vorhanden, aber echt erscheinend\grqq \cite{DudenVirtuell} beschrieben. Ein Beispiel hierfür ist der virtuelle Speicher. Ein virtueller Speicher ist physisch nicht vorhanden, funktioniert aus Benutzersicht jedoch identisch wie ein physischer Speicher. Im Duden hat Realität unter anderem die Synonyme Wirklichkeit, Ernstfall und Leben \cite{DudenRealitaet}. Die Realität ist demzufolge \glqq die reale Welt, in die ein Mensch hineingeboren und die von ihm durch seine Sinneseindrücke wahrgenommen wird\grqq \cite{omnia.2017}.

Die \autoref{fig:mixed-reality} zeigt die Abgrenzung zwischen \ac{VR}, \ac{AR} und \ac{MR}. Während \ac{VR} einer virtuellen Umwelt dargestellt wird, wird bei \ac{AR} die reale Umwelt durch virtuelle Informationen erweitert. \ac{AR} sollte vielen nicht unbekannt sein, da \ac{AR} unter anderem beim Fußball bei einem Freistoß im TV-Bild angewendet wird. Im Bild wird eine Linie inklusive der Entfernung bis zum Tor dargestellt. \ac{MR} hingegen ist eine Technik, die sich sowohl aus \ac{AR} als auch aus \ac{VR} zusammensetzt. Sie vermischt eine reale Wahrnehmung mit einer künstlich erzeugten Wahrnehmung. \ac{MR} kann sowohl auf einer virtuellen Umgebung basieren, in die reale Informationen hinzugefügt werden, oder auf der realen Umgebung, in die virtuelle Objekte hinzugefügt werden, die fest in der Umgebung verankert werden und mit denen interagiert werden kann. \ac{MR} lässt sich mithilfe der von Microsoft entwickelten Datenbrille Hololens erleben. Mithilfe dieser Datenbrille wird die Umgebung um virtuelle Objekte erweitert, mit denen der Benutzer interagieren kann.
\begin{figure}[!htbp]
	\centering
	\includegraphics[width=1\linewidth]{mixed-reality}
	\caption[Abgrenzung VR, AR und MR]{Abgrenzung VR, AR und MR \cite[S. 20]{BurofurTechnikfolgenAbschatzungbeimDeutschenBundestag.2019}}
	\label{fig:mixed-reality}
\end{figure}

\section{\acs{VR}-Headset}
\todo[inline]{VR-Headset oder VR-System?????}
\todo[inline]{Systemvoraussetzungen mit rein nehmen????}
Zur Umsetzung von \ac{VR} in der Praxis wird ein \ac{VR}-fähiges System benötigt, welches möglichst viele Sinne des Benutzers stimuliert \cite{Dorner.2019}. Nach \citeauthor{DoernerWahrnehmung} sind die wichtigsten Sinne im \ac{VR} Umfeld der visuelle, der akustische sowie der haptische Sinn \cite{DoernerWahrnehmung}. Im Rahmen dieser Arbeit wird das \acs{VR} System HTC Vive verwendet. Das System wurde von HTC gemeinsam mit dem Spieleentwickler Valve entwickelt und wurde im April 2016 auf den Markt gebracht \cite{Fehrenbach.14.4.2016}. Das System setzt sich aus den Komponenten \ac{HMD}, Controller und Tracking System zusammen. Die einzelnen Komponenten werden nachfolgend erläutert.

\subsection{\acl{HMD}}
Da der visuelle Sinn \glqq sicherlich die wichtigste Informationsquelle bei der Wahrnehmung\grqq \cite{DoernerWahrnehmung} von \ac{VR}-Umgebungen ist, werden \ac{HMD} zum Eintauchen in eine \ac{VR}-Umgebung verwendet. Die Übersetzung von \ac{HMD} lautet \glqq ein am Kopf montiertes Display\grqq. Wie die Übersetzung nahelegt, wird ein Display am Kopf des Benutzers montiert. Ein \ac{HMD} ist ein immersives Display, welches den Sichtbereich des Benutzers von der kompletten Umgebung abschirmt. Die Immersion bei einem \ac{HMD} hängt von der Größe des Sichtfeldes ab. Während ein großes Sichtfeld als immersiv gilt, ist ein kleines weniger immersiv. \cite{Dorner.2019} Jedes Auge hat sein eigenes Display. Da ein \ac{HMD} wie eine Brille vor den Augen angebracht ist und komplett am Kopf befestigt wird, wird ein \ac{VR}-System auch als \ac{VR}-Brille oder \ac{VR}-Headset bezeichnet.

In \autoref{fig:vive-hardware-hmd-1} ist das \ac{HMD} der HTC Vive abgebildet. Das Headset wird mittels Klettverschluss am Kopf befestigt. Die HTC Vive verwendet AMOLED Displays mit einer Auflösung von insgesamt \mbox{2160 x 1200} Pixel, dies entspricht einer Auflösung von \mbox{1080 x 1200} Pixel pro Auge. Der Sichtbereich beträgt 110°, die Bildwiederholfrequenz 90 Hz. Es ist möglich Einstellungen bei der Pupillendistanz und des Objektivabstands vorzunehmen. Das Headset ist über Kabel mit dem Computer verbunden. \cite{ViveProduct}

\begin{figure}[!htbp]
	\centering
	\includegraphics[width=0.5\linewidth]{vive-hardware-hmd-1}
	\caption[HMD der HTC Vive]{\acs{HMD} der HTC Vive \cite{ViveHMD}}
	\label{fig:vive-hardware-hmd-1}
\end{figure}

\subsection{Controller}
Die HTC Vive beinhaltet die in \autoref{fig:vive-hardware-controllers-1} dargestellten Controller, welche für die Interaktion in \ac{VR} verwendet werden können. Durch die Controller ist der Benutzer nicht von Maus und Tastatur abhängig, um mit der \ac{VR}-Umgebung interagieren zu können. Die Controller unterstützen haptisches Feedback mit einer hohen Auflösung, wodurch das \ac{VR}-Erlebnis durch den haptischen Sinn erweitert wird. Die Controller können den Benutzer bei der Bewegung durch Teleportation in \ac{VR} unterstützen. Auf der Oberseite der Controller befindet sich ein Multifunktions-Trackpad, ein Systemknopf, eine Menütaste sowie Greifknöpfe. Auf der Unterseite des Controllers befindet sich ein zweistufiger Abzug. \cite{ViveProduct}

\begin{figure}[!htbp]
	\centering
	\includegraphics[width=0.5\linewidth]{vive-hardware-controllers-1}
	\caption[Controller der HTC Vive]{Controller der HTC Vive \cite{ViveControllers}}
	\label{fig:vive-hardware-controllers-1}
\end{figure}

\subsection{Tracking System}
Für das Umsehen in einer \ac{VR}-Umgebung, wird ein Tracking System benötigt. Ein Tracking System erfasst die Position und die Bewegungen des Benutzers. Nach \citeauthor{Sauter.2015} unterschied sich die HTC Vive nach der Vorstellung auf der Mobile World Congress insbesondere durch das verwendete Tracking. Die Präsenz in \ac{VR} war weitaus besser als das anderer Hersteller. Während der Benutzer bei anderen Herstellern in der \ac{VR}-Umgebung stand und zuschaute, konnte der Benutzer mithilfe des Trackings der HTC Vive erstmals frei bewegen und mit der Umgebung interagieren. \cite{Sauter.2015} 

Das verwendetet Tracking System ist das von der Firma Valve entwickelte SteamVR-Tracking. Valve beschreibt SteamVR-Tracking als eine Hardware- und Softwarelösung, durch die Geräte in Echtzeit ihre Position im Raum ermitteln \cite{Valve.2020}. Während zum Beispiel bei Oculus VR kleine Infrarot-LEDs auf dem \ac{HMD} sitzen, welche von einer externen Kamera erfasst werden, verwendet SteamVR-Tracking keine Kameras \cite{Sauter.2015}. Stattdessen werden zwei Basisstationen verwendet (siehe \autoref{fig:vive-hardware-base-stations}), welche mithilfe von zwei Laserstrahlen der Klasse 1 \glqq das Zimmer mit mehreren Synchronimpulsen und Laserstrahlen im Radius von bis zu 5 Metern\grqq \cite{Valve.2020} scannen. Je ein Laser wird horizontal und vertikal rotiert. Daher wird das System auch als Lighthouse (Leuchtturm) System bezeichnet. In den zu erfassenden Objekten (zum Beispiel \ac{HMD}, Controller, etc.) sind Fotowiderstände als Sensoren integriert. Die millimetergenaue Position der Objekte wird mithilfe der integrierten Sensoren durch das Objekt selbst bestimmt. \cite{Yates.20160512}

\begin{figure}[!htbp]
	\centering
	\includegraphics[width=0.5\linewidth]{vive-hardware-base-stations}
	\caption[Lighthouse Basisstationen]{Lighthouse Basisstationen \cite{ViveBaseStation}}
	\label{fig:vive-hardware-base-stations}
\end{figure}

In \autoref{fig:graphic-tracking-steamvr} ist die Funktionsweise von SteamVR-Tracking grafisch dargestellt. Die zwei Basisstationen werden gegenüberliegend an der Wand montiert. Dies ermöglicht eine 360°-Abdeckung des Raumes. Das blaue Rechteck auf dem Boden definiert den Spielbereich, in dem sich der Benutzer frei bewegen kann. Dieser Bereich wird vom Benutzer kalibriert. Die Größe ist auf maximal 15 m$^2$ begrenzt \cite{ViveProduct}. Um ein unabsichtliches Verlassen des Spielbereichs des Benutzers zu verhindern, wird der Benutzer durch das Einblenden einer Chaperone-Spielbereichsbegrenzung in Form eines Energiefeldes bei der Grenze gewarnt \cite{ViveProduct}. 

\begin{figure}[!htbp]
	\centering
	\includegraphics[width=1\linewidth]{graphic-tracking-steamvr}
	\caption[Steam VR scannt mit Lighthouse]{Steam VR scannt mit Lighthouse \cite{Sauter.2015}}
	\label{fig:graphic-tracking-steamvr}
\end{figure}

\subsection{SteamVR}
\todo[inline]{SteamVR wurde von Valve und HTC als Unterstützung für Entwickler – zusammen mit dem HTC Vive System – veröffentlicht. Es bietet eine Reihe von API-Aufrufen und vorgefertigten Spiele Objekten, welche die Entwicklung von VR-Anwendungen zugänglicher gestaltet. Die Nutzung von SteamVR ist frei unter ihrer Steamworks VR API Code Lizenz}
\todo[inline]{Überhaupt was zu schreiben?? Oder erst in Unity?? Iwo muss der Zugriff für die Entwickler beschrieben werden.}

\section{Eyetracking}
Eyetracking ist eine Technologie, die erkennt, in welche Richtung eine Person ihren Blick richtet. Hierfür werden beim Eyetracking Blick sowie Augenbewegungen erfasst. Die hauptsächlichen Parameter, die durch das Eyetracking erfasst werden, sind Fixiationen (von dem Benutzer fixierter Punkt oder Objekt), Sakkaden (schnelle Augenbewegungen bei der Erfassung eines neuen Fixpunktes) und Regression (Rücksprung zu vorherigen Fixpunkte oder Objekte).\\
Im Rahmen dieser Arbeit wird ein Eyetracker von Pupil Labs verwendet. Sie entwickeln seit 2014 die Plattform Pupil Core, die aus einer Open-Soure-Suite sowie dem Eyetracker besteht. In \autoref{fig:pupil_labs_headset} ist das tragbare Eyetracker Headset Pupil Core zu sehen, welches wie eine Brille getragen wird. An dem Headset ist eine Blickfeldkamera (Nummer 1) angebracht, welche das Blickfeld des Benutzers aufnimmt. \todo{Infrarot} Mithilfe der Augenkameras (Nummer 3) lässt sich das komplette Auge erfassen. Die Augenkameras lassen sich individuell auf die Augen einstellen. Ein USB-C Kabel (Nummer 4) dient als Stromversorgung sowie für den Austausch der Videodaten der Kameras. Wird der Eyetracker mit dem Computer verbunden, dann lässt sich mit der Pupil Core Software das Eyetracking starten. Mithilfe der Software lassen sich die Eyetracking-Daten aus den Videostreams auslesen, auswerten und über eine Netzwerk Schnittstelle zur Verfügung stellen. Zudem kann in das Umgebungsvideo der Blickfeldkamera der Punkt angezeigt werden, auf den der Benutzer seinen Blick fixiert. \\

\begin{figure}[!htbp]
	\centering
	\includegraphics[width=0.75\linewidth]{pupil_labs_headset}
	\caption[Pupil Core Headset]{Pupil Core Headset \cite{PupilLabsHW}}
	\label{fig:pupil_labs_headset}
\end{figure}

Da in dieser Arbeit das Eyetracking innerhalb einer \ac{VR}-Umgebung untersucht werden soll, wird ein speziell für die HTC Vive entwickelter Eyetracker von Pupil Labs verwendet. Dies ist ein Add-on von Pupil Labs, welches in das HTC Vive Headset eingebaut wird und während dem Tragen des Headsets verwendet werden kann. In \autoref{fig:pupil_labs_addon} ist das Add-on von Pupil Labs und das \ac{VR}-Headset HTC Vive dargestellt. Das Add-on wird in die Innenseite um die Linsen des Headsets angebracht. 

\begin{figure}[!htbp]
	\centering
	\includegraphics[width=0.5\linewidth]{vive-addon}
	\caption[Pupil Core Add-on für HTC Vive]{Pupil Core Add-on für HTC Vive \cite{PupilLabsAddOn}}
	\label{fig:pupil_labs_addon}
\end{figure}

\todo{Zugriff auf Daten}
\cite{Clay_Koenig_Koenig_2019}

\cite{Kassner_2014}

\section{Unity}
Standardmäßig entspricht 1 Einheit 1 Meter (siehe \cite{BrentAllard.2017} \& \cite{AVividLight.2010}) --> Aussage ist plausibel, da standardmäßig in den Projekteinstellungen bei Physics die Gravitation auf -9,81 gesetzt ist.

\subsection{steamVR}

\subsection{hmd\_eyes}

\section{Fitts' Law}
\todo[inline]{Fitts's Law muss hier noch rein}