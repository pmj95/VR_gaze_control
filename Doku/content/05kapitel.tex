%!TEX root = ../dokumentation.tex

\chapter{Durchführung  der Probandenversuche} 
\todo[inline]{Erfassung der Probandendaten (sehhilfe, Alter, etc)}
\todo[inline]{Gaze Data Logging hier und Kap 4}

Im folgenden Kapitel werden die geplanten Probandenversuche und deren Durchführung genauer erklärt. Während der Bearbeitung der Arbeit hat sich die grundlegende Situation jedoch so sehr geändert, dass eine Durchführung, wie geplant, nicht stattfinden konnte. Mehr hierzu in Kapitel \ref{section:corona}.
\section{Probleme der Probandenversuche}
\label{section:corona}
Die Durchführung dieser Arbeit war für den Zeitraum November 2019 bis Juni 2020 vorgesehen. Die Durchführung der Probandenversuche war in der ursprünglichen Planung für April 2020 angesetzt. Allerdings wurden durch die Corona-Pandemie \todo{offizielle Quelle zur aktuellen Situation} Kontaktbeschränkungen beschlossen, die eine Durchführung in geplanter Größenordnung verbieten. Ursprünglich war mit einer Teilnehmerzahl von rund 20 Probanden geplant worden. Leider war es nicht möglich, in dieser Größenordnung den Versuch durchzuführen. Alle Daten, die erhoben wurden, sind von einem Probanden und den Autoren selbst erstellt. Dies beeinflusst die Messergebnisse enorm, sodass keine klaren Rückschlüsse und somit keine eindeutige Antwort auf die Forschungsfrage gegeben werden kann. Das folgende Kapitel wird unter Berücksichtigung dieser Einschränkung die vorhandenen Daten auswerten und anhand derer probieren, ob erste Aussagen über die Eignung von Eye-Tracking in VR möglich sind.
\section{Versuchsaufbau} 
In Kapitel \ref{section:versuchsumgebung} wird der Aufbau der unterschiedlichen Level bereits erklärt. Es ergeben sich folgende Kombinationsmöglichkeiten für einen Versuchsaufbau:
\begin{itemize}
	\item 4 verschiedene Umgebungen
	\item 2 unterschiedliche Formen des Anvisierens (Laserpointer/Controller und Eye-Tracking)
	\item 2 unterschiedliche Formen des Auswählens (Triggerbetätigung und Blinzeln)
	\item 3 unterschiedliche Größen für die Knöpfe
\end{itemize}
Daraus lässt sich berechnen wie viele verschiedene Aufbauten A es insgesamt gibt:
\begin{align}
	A=4*2*2*3=48
\end{align}
Um erste Indizien geben zu können, sollen für jeden möglichen Versuchsaufbau 5 Messungen zur Verfügung stehen. Das bedeutet, dass insgesamt mindestens 240 Messungen durchgeführt werden sollten. Dabei soll jeder Proband alle Steuermöglichkeiten gleich oft testen. Das bedeutet, dass bei 20 Probanden jeder Proband 12 Messungen durchführt. Jeweils 3 sind eine Steuermöglichkeits-Kombination. Diese drei Versuche werden nach folgendem Muster entworfen:
\begin{enumerate}
	\item Das erste Level ist eine der vier Steuermöglichkeiten mit einer zugeteilten Distanz und einer zugeteilten Knopfgröße.
	\item Das zweite Level verändert die Distanz zu den Objekten, behält aber die Knopfgröße und die Steuerart bei.
	\item Das dritte Level verändert die Größe des Knopfes, behält aber Steuerart und Distanz vom ersten Level bei.
	\item Dieses Muster wiederholt sich vier mal - für jede Steuerart ein mal.
\end{enumerate}
Mit dieser Methode ist es möglich, alle unterschiedlichen Parameter möglichst gleichmäßig zu testen. Da jeder Proband einen unterschiedlichen Ablauf hat, werden Lerneffekte und andere unerwünschte Nebeneffekte neutralisiert. Mit einem Plan, wie in Anhang \todo{Anhang mit Tabelle für Probanden und Versuche aufgelistet} beschrieben, würde jeder Testaufbau mit 5 Versuchen getestet werden. Durch die Veränderung eines Einzelnen Parameters zwischen jedem Versuch lässt sich der Einfluss jedes Parameters in mehreren Testaufbauten messen. Im Anhang \ref{table:probanden} ist die verwendete Testreihe gezeigt. Jede Zeile beschreibt einen Probanden. Die Anordnung der Tabelle ist zufällig generiert und hilft so, mögliche Lerneffekte oder ähnliches zu verhindern. Die Daten in der Tabelle geben für jeden der 20 Probanden 12 Versuche an. Ein Versuch ist als Tupel von zwei Zahlen angegeben. Es werden immer drei Versuche pro Proband gemacht. Ein Tupel (x, y) beschreibt als x die Größe der Knöpfe und als y das gewählte Level. Der Parameter x hat drei Stufen, eins bis drei, wobei eins der größten Größe entspricht und drei der kleinsten. Bei y beschreiben die Werte eins bis drei die unterschiedliche Distanz zu den Objekten, bzw. das dementsprechende Level. Bei y=1 sind die Knöpfe sehr nah, bei y=3 sind sie weit weg. Bei y=4 wird das 3D-Level genutzt. Hier variiert die Distanz der Knöpfe je nach Knopf. 

\subsection{Ablauf eines Probandenversuchs}
Der Probandenversuch findet in Gruppenterminen statt. Dabei werden nach und nach Probanden benötigt. Da wegen Hardwareeinschränkungen nicht zeitgleich getestet werden kann, wird ein Zeitplan entworfen. Jeder Proband kommt einzeln in den Testraum. Hier werden zunächst Informationen gesammelt, wie bspw. das Alter, Erfahrung mit VR und allgemeiner Umgang mit Computern (Kenntnisse sehr gut bis schlecht) \todo{umformulieren}. Daraufhin wird dem Nutzer die VR-Brille aufgesetzt und die grundsätzlichen Einstellungen (z.B. Band am Kopf anpassen) werden vorgenommen. In einer Beispielumgebung wird dem Proband das Prinzip des Versuchs erklärt. Die Augen werden für das Eye-Tracking kalibriert. Wenn der Proband sich bereit für den Versuch fühlt, werden nach und nach die entsprechenden Level geladen und bewältigt. Die Messdaten werden automatisch erfasst und zur Auswertung gespeichert. Die Daten und Auswertung sind hier zu finden \todo{daten und auswertung verlinken und hinzufügen}. Sobald der Proband die für ihn bestimmten zwölf Versuche durchgeführt hat, füllt er einen Fragebogen (siehe Kapitel \ref{section:fragebogen}) aus. Die Ergebnisse und Auswertung dieser sind hier zu finden \todo{daten und auswertung verlinken und hinzufügen}.

\subsection{Teilnehmer der Studie}
\todo[inline]{bei anderen studien schauen wie genua die sowas beschrieben haben}
Jörn - 24, Brille (in VR ohne Brille), Einige Erfahrungen mit VR, Entwickler des Projekts
Clemens - 21, Kontaktlinsen, Regelmäßiger VR-nutzer, Entwickler des Projekts
Julian - 21, keine Sehhilfe, Einige Erfahrungen mit VR, Erster Kontakt mit Projekt
Andre - 20, keine Sehhilfe, Regelmäßiger VR-Nutzer, Erster Kontakt mit Projekt

\section{Fragebogen} 
\label{section:fragebogen}
\todo[inline]{Anhang und Auswertung}
Der Fragebogen stellt einen wichtigen Teil in der Beantwortung der Forschungsfrage dar. Da es das Ziel dieser Arbeit ist, herauszufinden, ob sich Eye-Tracking für die Steuerung in VR eignet, ist das subjektive Empfinden der Nutzer enorm wichtig. Mit den Messwerten aus den Versuchen lassen sich zwar Parameter wie Schnelligkeit vergleichen, allerdings ist die subjektive Empfindung bei Eingabegeräten immer ein wichtiger Faktor. Deshalb soll der Fragebogen die Frage beantworten, welche Steuermöglichkeit von den Probanden bevorzugt wird und warum das so ist.
\subsection{Fragen}
Der Fragebogen besteht aus zwei Teilen: Einerseits werden Fragen zu jeder Steuermöglichkeit separat gefragt, andererseits werden Fragen zum allgemeinen Empfinden und den Steuermöglichkeiten im Vergleich gefragt.
Zu jeder Steuermöglichkeit wird gefragt, :
\begin{itemize}
	\item wie leicht die Fixierung und das Auswählen des richtigen Knopfes gefallen ist.
	\item wie gut man sich eine Steuerung mit dieser Konfiguration vorstellen kann.
	\item wie stark eine Veränderung der Distanz und der Größe der Knöpfe einen Einfluss auf die Steuerung hat.
\end{itemize}
Am Ende der Befragung wird der Proband gebeten, ein begründetes Ranking der Steuermöglichkeiten abzugeben. Hier können auch offene Punkte angemerkt werden, die nicht im restlichen Teil des Fragebogens abgefragt werden.


\section{Ergebnisse}
\subsection{Eye-Tracking im Vergleich}
\subsection{Kombinationen der Steuermöglichkeiten}
\subsection{Einfluss der Veränderung der Knopfgröße}
\subsection{Einfluss der Veränderung der Distanz}
\section{Ergebnisse des Fragebogens}
Im Folgenden werden die Ergebnisse des Fragebogens dargestellt. Da jedoch eine Veränderung der Distanz aufgrund der COVID-19-Pandemie nicht untersucht wurde, wurde diese Frage im Fragebogen übergangen. 

Im gesamten Ranking der Steuermöglichkeiten waren sich die Probanden grundsätzlich einig. Alle vier Probanden haben die Steuermethode Laserpointer mit Trigger als die beste und Eye-Tracking mit Blinzeln als die schlechteste Steuermethode angegeben. Bei drei der vier Probanden hat die Steuermethode Laserpointer mit Blinzeln den zweiten und Eye-Tracking mit Trigger den dritten Platz belegt. Bei einem Probanden waren diese beiden Steuermethoden vertauscht. In den folgenden Unterabschnitten werden die Ergebnisse zu den einzelnen Steuermöglichkeiten dargestellt und die Kommentare der Probanden aufgenommen. Die genauen Angaben der Probanden sind im Anhang \todo{link zu anhang} zu finden.
\subsection{Laserpointer mit Trigger}
Alle vier Probanden konnten den Sätzen "Es fiel mir leicht, die richtigen Blöcke zu fixieren/auszuwählen" und "Ich kann mir eine Menüsteuerung mit dieser Methode vorstellen" komplett zustimmen. Eine Veränderung der Knopfgröße hatte für alle vier Probanden "ein wenig" Einfluss auf die Fähigkeit, Knöpfe anzuvisieren und auszuwählen. Als Begründung für die erste Position im Ranking wurden vor allem Intuitivität und die einfache Gewöhnung an diese Steuermethode genannt. Den Nutzern ist sowohl das Anvisieren, als auch das Auswählen leicht gefallen. 
\subsection{Laserpointer mit Blinzeln}
Diese Methode hat im Ranking bei drei von vier Probanden den zweiten Platz belegt. Dem Satz "Es fiel mir leicht, die richtigen Blöcke zu fixieren" konnten alle vier Probanden komplett zustimmen. Bei der Auswahl der Blöcke sind die Antworten sehr unterschiedlich ausgefallen. Der Proband, der diese Methode auf Rang 3 platziert hat, konnte der Aussage "Es fiel mir leicht, die richtigen Blöcke zu auszuwählen"  nicht zustimmen. Die anderen Probanden konnten der Aussage größtenteils (2x) oder komplett (1x) zustimmen. Der Aussage \grqq Ich kann mir eine Menüsteuerung mit dieser Methode vorstellen\grqq konnte ein Proband komplett zustimmen, zwei Probanden konnten größtenteils zustimmen und der Proband, der Probleme mit dem Blinzeln zum Auswählen hatte konnte dieser Aussage gar nicht zustimmen. Eine Veränderung der Knopfgröße hatte für drei der vier Probanden einen geringen Einfluss, ein Proband stufte den Einfluss als normal ein. Die Begründung der drei Probanden, die diese Methode auf den zweiten Rang gesetzt hatten, nannten vor allem die gute Genauigkeit des Laserpointers als Grund für die hohe Platzierung. Das Blinzeln wurde zwar meist als gute Bestätigungsmethode genannt, einem Proband fiel es aber schwer, nicht aus Versehen an der falschen Stelle zu blinzeln. 
\subsection{Eye-Tracking mit Trigger}
Beim Einsatz von Eye-Tracking in Verbindung mit dem Trigger des Vive Controllers konnten keiner der Probanden dem Satz \grqq Es fiel mir leicht, die richtigen Blöcke zu fixieren\grqq komplett zustimmen. Zwei der Probanden gaben an, der Aussage größtenteils zustimmen zu können, einer konnte teils zustimmen, einer nur ein wenig. Der Aussage \grqq Es fiel mir leicht, die richtigen Blöcke auszuwählen\grqq konnten alle Probanden größtenteils zustimmen. Zwei der Probanden konnten der Aussage \grqq Ich kann mir eine Menüsteuerung mit dieser Methode vorstellen\grqq komplett zustimmen, einer konnte größtenteils und einer teils zustimmen. Den Einfluss durch eine Veränderung der Knopfgröße beschrieben den Probanden als normal (1x) bis stark (3x). Ein Proband hat diese Methode als zweitbeste bewertet.
\subsection{Eye-Tracking und Blinzeln}
Diese Methode hat im Vergleich zu den anderen Methoden mit Abstand am schlechtesten abgeschnitten. Keinem der Probanden fiel es leicht die Blöcke anzuvisieren oder auszuwählen. Keiner der Probanden kann sich diese Steuermethode als Menüsteuerung vorstellen. Eine Veränderung der Knopfgröße hatte bei allen Probanden einen sehr starken Einfluss. Alle Probanden hatten das Problem, dass beim Blinzeln zum Bestätigen eines Knopfes der Eye-Tracker wieder verrutscht ist und somit keine genaue Steuerung möglich war.
\subsubsection{Vergleich mit numerischen Werten}
