%!TEX root = ../dokumentation.tex

\chapter{Fazit}
Mit den Probandenversuchen wurde gezeigt, dass Eye-Tracking in Verbindung mit einem Trigger eine valide Steuermethode ist. Zwar ist es nicht die effizienteste Methode, allerdings ist einerseits eine Kombination aus Controllern und Eye-Tracking denkbar, andererseits erlaubt Eye-Tracking auch körperlich benachteiligten Menschen eine intuitive Steuerung in VR. Durch die anderen Einsatzgebiete von Eye-Tracking in VR, wie etwa Foveated Rendering, ist das Verwenden der Technik ein vielversprechender Ansatz für die Zukunft. Wir sind gespannt, welche Möglichkeiten sich in naher Zukunft mit dem Einbau von Eye-Tracking Hardware in VR-Systemen noch ergeben werden.

\section{kritische Würdigung}
Vor allem in Anbetracht der Studie von \citeauthor{Pai.2019} und \citeauthor{D.Kumar.2016} sind die Ergebnisse dieser Studie interessant, da Eye-Tracking sich in allen drei Untersuchungen als valide Steuermethode gezeigt hat. Die Ergebnisse der Forscher können hier also bestätigt werden. In den Details der Ergebnisse zeigen sich jedoch Unterschiede. Dies ist vor allem auf den unterschiedlichen Versuchsaufbau, aber auch auf die limitierten Forschungsmöglichkeiten durch die COVID-19-Pandemie zurückzuführen. Die Untersuchung der Zeiten zwischen dem Aktivieren von zwei Knöpfen hat außerdem gezeigt, dass der Versuchsaufbau in der jetzigen Form teilweise verzerrte Daten liefern kann. Im folgenden Abschnitt werden einige Verbesserungsmöglichkeiten für den Versuch angeführt. 

\section{Ausblick und Anpassungen für nachfolgende Forschungen}
Durch die COVID-19-Pandemie ist eine vollständige Durchführung nicht möglich gewesen. Eine Durchführung dieser wäre jedoch von hohem Interesse, da einige Messungenauigkeiten, die in \autoref{section:discussion} beschrieben wurden, so bereinigt werden können. Während der Bearbeitung und Durchführung der Probandenversuche sind einige mögliche Verbesserungen für eine zukünftige, auf dieser Arbeit basierende, Forschung aufgefallen. Diese werden in diesem Abschnitt genauer erläutert. 
\subsection{Anpassung des Leveldesigns}
Das Leveldesign hat sich grundsätzlich als gut bewiesen, hat aber auch Schwachstellen. So ist es bei den Eye-Tracking Versuchen ungünstig für die Bewertung nach Fitts Gesetz, dass der Nutzer seinen Blick immer wieder nach oben auf das Informationsfeld richten muss. In einem zukünftigen Versuchsaufbau kann dies beispielsweise mit einem Aufleuchten des auszuwählenden Knopfes umgangen werden. Auch könnte es sinnvoll sein, die komplette Zahlenfolge von Beginn an anzuzeigen, um dem Nutzer die Möglichkeit zu geben, die Zahlenfolge noch schneller abzuarbeiten. Ein weiterer Ansatz wäre es, nur den auszuwählenden Knopf anzuzeigen, wodurch dem Nutzer sofort ersichtlich wird, welcher Knopf als näcshtes zu drücken ist.

\subsection{Blinzeln als Bestätigungsmethode}
Diese Arbeit zeigt sehr eindeutig, dass Blinzeln als Eingabemethode in Verbindung mit Eye-Tracking nicht wirklich nutzbar ist. Eine Durchführung einer ähnlichen Untersuchung mit mehr Probanden und mehr Versuchen pro Proband kann die Zeit pro Proband deutlich reduzieren, indem diese Eingabemethode nicht mehr untersucht wird. Stattdessen kann eine weitere Steuervariation eingebracht werden, um einen noch besseren Vergleich ziehen zu können. 

\subsection{Verminderung der Lerneffekte}
Wie in den ursprünglich geplanten Probandenversuchen sollte eine zufällige Reihenfolge der unterschiedlichen Versuche stattfinden, um mögliche Lerneffekte ausgleichen zu können. Dadurch werden die gemessenen Werte zuverlässiger und aussagekräftiger. Jedoch ist eine gewisse Eingewöhnung an die Steuerung nicht unbedingt schlecht, da dadurch bessere Aussagen über die Alltagsfähigkeit einer solchen Steuerung getroffen werden können. 
